\nonstopmode{}
\documentclass[a4paper]{book}
\usepackage[times,inconsolata,hyper]{Rd}
\usepackage{makeidx}
\usepackage[utf8,latin1]{inputenc}
% \usepackage{graphicx} % @USE GRAPHICX@
\makeindex{}
\begin{document}
\chapter*{}
\begin{center}
{\textbf{\huge Package `DriftR'}}
\par\bigskip{\large \today}
\end{center}
\begin{description}
\raggedright{}
\item[Type]\AsIs{Package}
\item[Title]\AsIs{What the package does (short line)}
\item[Version]\AsIs{1.0}
\item[Date]\AsIs{2014-09-27}
\item[Author]\AsIs{Who wrote it}
\item[Maintainer]\AsIs{Who to complain to }\email{yourfault@somewhere.net}\AsIs{}
\item[Description]\AsIs{More about what it does (maybe more than one line)}
\item[License]\AsIs{What license is it under?}
\item[Depends]\AsIs{tidyr, dplyr, ggplot2}
\item[Imports]\AsIs{Rcpp (>= 0.11.2), RcppArmadillo (>= 0.4.400.0)}
\item[LinkingTo]\AsIs{Rcpp, RcppArmadillo}
\item[NeedsCompilation]\AsIs{yes}
\end{description}
\Rdcontents{\R{} topics documented:}
\inputencoding{utf8}
\HeaderA{DriftR-package}{What the package does (short line) \textasciitilde{}\textasciitilde{} package title \textasciitilde{}\textasciitilde{}}{DriftR.Rdash.package}
\aliasA{DriftR}{DriftR-package}{DriftR}
\keyword{package}{DriftR-package}
%
\begin{Description}\relax
More about what it does (maybe more than one line)
\textasciitilde{}\textasciitilde{} A concise (1-5 lines) description of the package \textasciitilde{}\textasciitilde{}
\end{Description}
%
\begin{Details}\relax

\Tabular{ll}{
Package: & DriftR\\{}
Type: & Package\\{}
Version: & 1.0\\{}
Date: & 2014-09-27\\{}
License: & What license is it under?\\{}
}
\textasciitilde{}\textasciitilde{} An overview of how to use the package, including the most important \textasciitilde{}\textasciitilde{}
\textasciitilde{}\textasciitilde{} functions \textasciitilde{}\textasciitilde{}
\end{Details}
%
\begin{Author}\relax
Who wrote it

Maintainer: Who to complain to <yourfault@somewhere.net>
\textasciitilde{}\textasciitilde{} The author and/or maintainer of the package \textasciitilde{}\textasciitilde{}
\end{Author}
%
\begin{References}\relax
\textasciitilde{}\textasciitilde{} Literature or other references for background information \textasciitilde{}\textasciitilde{}
\end{References}
%
\begin{SeeAlso}\relax
\textasciitilde{}\textasciitilde{} Optional links to other man pages, e.g. \textasciitilde{}\textasciitilde{}
\textasciitilde{}\textasciitilde{} \code{\LinkA{<pkg>}{<pkg>}} \textasciitilde{}\textasciitilde{}
\end{SeeAlso}
%
\begin{Examples}
\begin{ExampleCode}
~~ simple examples of the most important functions ~~
\end{ExampleCode}
\end{Examples}
\inputencoding{utf8}
\HeaderA{addMissingVals}{Add clones that haven't been counted If clones haven't been counted they don't appear rather than count as zero}{addMissingVals}
%
\begin{Description}\relax
Add clones that haven't been counted
If clones haven't been counted they don't appear rather than
count as zero
\end{Description}
%
\begin{Usage}
\begin{verbatim}
addMissingVals(data_Nths, frac_size)
\end{verbatim}
\end{Usage}
\inputencoding{utf8}
\HeaderA{format\_exp\_data}{Calculate proportions with 95\% credible interval (confidence interval). If there are replicates they will be pooled(!). It will work for both count table format and list format.}{format.Rul.exp.Rul.data}
%
\begin{Description}\relax
Calculate proportions with 95\% credible interval (confidence interval).
If there are replicates they will be pooled(!). It will work for both count table format and list format.
\end{Description}
%
\begin{Usage}
\begin{verbatim}
format_exp_data(data_x, time_points = NULL, clone_fractions = NULL)
\end{verbatim}
\end{Usage}
\inputencoding{utf8}
\HeaderA{format\_list2tbl}{Transform list format to table format with clone size along rows and time points along cols. Will return time points as colnames. This is the format used for inference If a clone size does not appear in any data point it will not be included (!!!)}{format.Rul.list2tbl}
%
\begin{Description}\relax
Transform list format to table format with clone size along rows and time points
along cols. Will return time points as colnames. This is the format used for inference
If a clone size does not appear in any data point it will not be included (!!!)
\end{Description}
%
\begin{Usage}
\begin{verbatim}
format_list2tbl(list_x, condition_filter, clone_fractions = NULL)
\end{verbatim}
\end{Usage}
\inputencoding{utf8}
\HeaderA{format\_tbl2list}{Transform tbl format to list format. Useful for ggplot plotting}{format.Rul.tbl2list}
%
\begin{Description}\relax
Transform tbl format to list format. Useful for ggplot plotting
\end{Description}
%
\begin{Usage}
\begin{verbatim}
format_tbl2list(tbl_x, time_points, condition = "Condition1")
\end{verbatim}
\end{Usage}
\printindex{}
\end{document}
